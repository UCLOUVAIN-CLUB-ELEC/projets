Pour la connexion des différentes parties découvertes lors des dernières séances, vous devriez être maintenant capables de vous en sortir avec les informations déjà reçues. Si vous avez la moindre question, les membres du Club se feront un plaisir de vous aider!\\

Jusqu'ici, vos circuits étaient alimentés par une tension de 5V, provenant soit d'une source fixe ou soit d'un câble USB. Pour rendre votre robot autonome dans ses déplacements, vous recevrez une pile 9V ainsi qu'un connecteur. Connectez les masses ensemble et la tension positive de la pile (fil rouge du connecteur) au port VIN de l'Arduino. Ce port permet de recevoir une tension d'alimentation entre 7V et 12V, qui est réduite à 5V par un régulateur interne de l'Arduino. Cette tension de 5V est utilisée par l'Arduino pour fonctionner et peut également être utilisée pour alimenter des circuits externes (par exemple le récepteur infrarouge) grâce à la sortie 5V. Comme le moteur consomme beaucoup de puissance, pour éviter de surcharger le régulateur de tension de l'Arduino, ne branchez pas l'alimentation $V_{DD,moteur}$ (cfr. hands-on 1) sur le 5V mais directement sur le 9V de la pile. Cela permettra également au moteur de tourner plus vite. ATTENTION! Ne branchez pas le 9V sur $V_{DD,controle}$, qui reste branché sur le 5V.\\

