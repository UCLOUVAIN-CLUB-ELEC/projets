\documentclass[a4paper,10pt,twoside]{article}
\usepackage[utf8]{inputenc}
\usepackage[french]{babel}
\usepackage[T1]{fontenc}
\usepackage{amsmath}
\usepackage{amsfonts}
\usepackage{amssymb}
\usepackage{graphicx}
\usepackage{multicol}
\usepackage{array}
\usepackage{float}
\usepackage{epstopdf}
\usepackage[justification=centering]{caption}
\usepackage{caption}
\usepackage{subfig}
\usepackage{gensymb}
\usepackage[bottom]{footmisc}
\usepackage{appendix}
\usepackage{pdfpages}
\usepackage{todonotes}
\usepackage{mathpazo}
\usepackage{titleps}
\usepackage{color}
\usepackage{hyperref}
\usepackage[skins]{tcolorbox}
\usepackage{sectsty}
\usepackage[arrowmos]{circuitikz}
\usepackage{pgfplots}
\usepackage{blindtext}
\usepackage{adjustbox}
\usepackage{listings}
\usepackage[inner=2.5cm,outer=2.5cm,top=3cm,bottom=3cm]{geometry}

\graphicspath{{pictures/}}
\setlength\parindent{0pt}
\renewcommand*\rmdefault{ppl}
\newcolumntype{C}[1]{>{\centering\let\newline\\\arraybackslash\hspace{0pt}}m{#1}}
\newcolumntype{R}[1]{>{\raggedright\arraybackslash}p{#1}}
\sectionfont{\large}
\subsectionfont{\normalsize}

% Page style definitions
\newpagestyle{main}{
	\sethead[Club ELEC : Hands-on 2][][]  % even
			{\chaptertitle}{}{Club ELEC : Hands-on 2}
	\headrule
    \setfoot[\thepage][][]
    		{}{}{\thepage}
}

\newpagestyle{appendix}{
	\sethead[Club ELEC : Annexes][][]  % even
			{}{}{Club ELEC : Annexes}
	\headrule
    \setfoot[\thepage][][]
    		{}{}{\thepage}
    \footrule
}

%----------------------------------------------------------------------------------------
%	TITLE SECTION
%----------------------------------------------------------------------------------------
\title{
	\vspace{2.5cm}
	\normalfont \normalsize
	\huge Club ELEC\\
	\vspace{2.5cm}
	\huge Projet Robot\\
	\vspace{.25cm}
	\Large HO2 - Télécommande infrarouges et programmation
	\vspace{2.5cm}
	\centering
}

\begin{document}
\renewcommand{\figurename}{Fig.}
\renewcommand{\thepage}{\roman{page}}
\setcounter{page}{1}

\pagenumbering{gobble}
\maketitle
\newpage
\pagenumbering{arabic}
\pagestyle{main}

\newpage
\null
\thispagestyle{empty}
\newpage
\clearpage

\setcounter{page}{1}

%%% Introduction
\section*{Introduction}
Pendant ce quadrimestre, le Club ELEC vous propose de développer un petit robot contrôlé par une télécommande à infrarouges. Pour ce faire, le développement du circuit se déroulera en 3 phases, chacune correspondant à une séance de hands-on proposée par le club.

\begin{itemize}
	\item[-] HO1: Contrôle des moteurs.
	\item[-] HO2: Télécommande infrarouges et programmation.
	\item[-] HO3: Assemblage du robot.
\end{itemize}

%%% Objectifs du HO1
\section*{Objectifs}
Les objectifs du premier hands-on sont:
\begin{itemize}
	\item[-] De se familiariser avec le matérial de base (breadboard, multimètre, oscilloscope) et les composants de base (résistances, capacités, amplificateurs opérationnels, composants intégrés) propres à l'électronique.
	\item[-] De comprendre le fonctionnement du circuit de contrôle du moteur, basé sur des signaux de contrôle adéquats et d'un pont H.
	\item[-] De connecter ce circuit à un moteur DC pour en vérifier le bon fonctionnement.
\end{itemize}

%%% Description du circuit
\section*{Installations nécessaires}
\input{installation.tex}

\clearpage
\newpage
%%% Montage du circuit
\section*{Interfaçage de composants autour de l'Arduino Nano}
Le circuit à implémenter pour connecter ce micro est donné à la Figure \ref{fig:circuit_H01}. Dans celui-ci, on retrouve la tension d'alimentation (\textsc{SUPPLY}) à 5V. Pour ce projet, nous utiliserons une alimentation réglable de laboratoire (si tu désires reproduire ce circuit chez toi, n'hésite pas à demander à un membre du staff quelle alimentation tu peux utiliser). Il te faudra également un micro et une résistance de 20k$\Omega$. 

\begin{figure}[!ht]
	\centering
	\includegraphics[width=.65\textwidth]{figures/circuit_1.png}
	\caption{Circuit HO1. $V1 = 5V$, $R1 = 20 k\Omega$}
	\label{fig:circuit_H01}
\end{figure}

Pour réaliser les connections, nous utiliserons une \textit{breadboard}. Ce sont des plaquettes qui permettent de facilement connecter et déconnecter des composants électroniques, bien pratiques pour effectuer des premiers tests. Pour l'utiliser, rien de plus simple: il suffit d'enfoncer les composants dans les trous de la plaquette en faisant en sorte de les placer correctement pour que les connections se fassent. Les trous de la plaquette sont connectés suivant le schéma de la Figure \ref{fig:breadboard}.

\begin{figure}[!ht]
	\centering
	\includegraphics[width=.5\textwidth]{figures/breadboard.PNG}
	\caption{breadboard}
	\label{fig:breadboard}
\end{figure}


%%% Programme pour récupérer les codes envoyés par la télécommande
\section*{Premier programme: récupération des codes envoyés par la télécommande}
Le but de cet exercice est d'établir une communication entre votre télécommande et l'Arduino. Typiquement, quand votre télécommande essaye de communiquer avec votre télévision, cela se passe 3 étapes:
\begin{enumerate}
  \item Quand vous appuyez sur un bouton, la télécommande envoie une valeur $X$
  via infrarouges.
  \item La télévision reçoit cette valeur $X$ via le récepteur infrarouges.
  \item En fonction de la valeur de $X$, le microcontrôleur de la télévision fait une action. Il peut
  soit augmenter le volume, changer de chaîne ou encore s'éteindre.
\end{enumerate}

La toute première étape est donc de savoir quelle valeur est envoyée par votre télécommande quand vous appuyez sur un certain bouton. Le programme \autoref{code1} permet de faire cela. En quelques mots, à chaque fois que l'Arduino reçoit un message infrarouges, il l'envoie directement à l'ordinateur. Cela vous permet donc de savoir le code correspondant à chacun des boutons de votre télécommande (notez ça sur un feuille ;)).

Pour voir les messages en console, vous devez aller dans \textit{Tools-->Serial Monitor}.

\lstset{language=C}
\begin{lstlisting}[frame=single,numbers=left,numberstyle=\small,label={code1},caption={Lecture de la télécommande}]
#include <IRremote.h>

const int RECV_PIN = 7;

IRrecv irrecv(RECV_PIN);
decode_results results;

void setup(){
  Serial.begin(9600);
  irrecv.enableIRIn();
}

void loop(){
  if (irrecv.decode(&results)){
        Serial.println(results.value, HEX);
        irrecv.resume();
  }
}
\end{lstlisting}


%%% Programme pour allumer/éteindre une LED lorsqu'on appuie sur une touche de la télécommande
\section*{Second programme: contrôle d'une LED avec la télécommande}
Une fois que vous avez identifié le code que votre télécommande envoie pour un certain bouton, vous pouvez adapter l’exemple \autoref{code2}.
Ici, à chaque itération de la fonction \textit{loop()}, l'Arduino regarde si le récepteur infrarouges a reçu une message. Pour chaque message reçu, il regarde à quel bouton ce code correspond (ici $AB456CD$ pour le bouton 1 et $05FBAC33$ pour le bouton 2). Ensuite, il exécute une fonction correspondant au bouton utilisé (\textit{IR\_button1()}).
Dans le cas du bouton 1, la Led s'allume ou s'éteint à chaque fois que le bouton est appuyé.

\lstset{language=C}
\begin{lstlisting}[frame=single,numbers=left,numberstyle=\small,label={code2},caption={Code 2}]
  #include <IRremote.h>

  #define REMOTE_BUTTON1    0xAB456CD   // Ligne a modifier
  #define REMOTE_BUTTON2    0x5FBAC33   // Ligne a modifier

  const int RECV_PIN = 7;
  const int LED_PIN = 13;

  IRrecv irrecv(RECV_PIN);
  decode_results results;

  int LED_val = 0;

  void setup() {
    Serial.begin(9600);
    pinMode(LED_PIN,OUTPUT);
    digitalWrite(LED_PIN,LED_val);
    irrecv.enableIRIn();
    delay(500);
  }

  void loop() {
    if (irrecv.decode(&results)){
      //Serial.println(results.value, HEX);
      irrecv.resume();
      switch (results.value) {
        case REMOTE_BUTTON1:
          IR_button1();
          break;
        case REMOTE_BUTTON2:
          IR_button2();
          break;
      }
      irrecv.resume();
      }
  }

  void IR_button1(){
      LED_val = 1-LED_val;
      digitalWrite(LED_PIN,LED_val);
  }

  void IR_button2(){
    // faire qqch
  }
\end{lstlisting}


%%% Contrôle du mouvement d'un robot
\section*{Contrôle du mouvement d'un robot}
Vous avez appris dans ce hands-on à contrôler des moteurs DC à partir de signaux de contrôle digitaux. Dans votre circuit, les valeurs de ces signaux de contrôle étaient fixées manuellement. Pour la suite de ce projet, vous recevrez, entre autres, un microcontrôleur de type Arduino qui permet d'envoyer des signaux de contrôle 5V, ainsi que 2 moteurs DC, un L293D et une pile 9V. Imaginez dès à présent la stratégie la plus efficace pour contrôler les mouvements de votre robot à partir de ce matériel. Voici quelques questions pour vous aiguiller:
\begin{itemize}
\item Quels sont les signaux à envoyer au L293D pour avancer? Et pour reculer?
\item Comment contrôler chacun des moteurs pour tourner?
\item Comment faire pour régler la vitesse du robot? 
\end{itemize}

%%%% Installations
%\section{Installations nécessaires}
%Avant de pouvoir programmer votre balise réceptrice, plusieurs étapes sont nécessaires.
%
%Il vous faut d'abord \textbf{télécharger et installer le programmateur Arduino}. Celui-ci est aussi appelé Arduino IDE (Arduino Integrated Development Environment). Voici les instructions à suivre :
%\begin{enumerate}
%	\item se rendre sur la page principale du site Arduino (\url{www.arduino.cc/}) ;
%	\item cliquer sur l'onglet "Software" ;
%	\item choisir votre version d'Arduino IDE (voir Fig. \ref{fig:arduino_ide}) et la télécharger ;
%	\item une fois le téléchargement effectué, il ne vous reste plus qu'à installer le programmateur.
%\end{enumerate}
%
%\begin{figure}[ht!]
%	\centering
%	\includegraphics[width=\textwidth]{imgs/arduino_ide.png}
%	\caption{Téléchargement du programmateur Arduino.}
%	\label{fig:arduino_ide}
%\end{figure}
%
%Patience, avant de programmer il vous reste encore quelques étapes. En effet, une fois l'IDE installé, il vous faudra également \textbf{installer la librairie RF24}. Celle-ci permettra à votre module Arduino Nano de communiquer correctement avec le module RF qui y est connecté . Voici les instructions à suivre :
%\begin{enumerate}
%	\item ouvrir l'Arduino IDE fraichement installé ;
%	\item aller dans \textit{Sketch --> Include Library --> Manage Libraries...} ;
%	\item taper "RF24" dans la barre de recherche ;
%	\item cliquer sur la librairie "RF24" (voir Fig. \ref{fig:arduino_RF24}) et l'installer ;
%	\item voilà !
%\end{enumerate}
%
%\begin{figure}[ht!]
%	\centering
%	\includegraphics[width=\textwidth]{imgs/arduino_RF24.png}
%	\caption{Installation de la librairie RF24.}
%	\label{fig:arduino_RF24}
%\end{figure}
%
%Vous êtes enfin prêts à programmer votre module Arduino. Vous l'aurez remarqué, Arduino est open-source et gratuit. Sachez aussi qu'il existe une floppée de librairies existantes pour faire à peu près n'importe quoi !
%
%%%% Principe de fonctionnement
%\section{Votre premier programme Arduino !}
%Pour vous aider à vous lancer dans le fabuleux monde de la programmation Arduino, voici quelques instructions pour lancer le code exemple pour votre baliser RF réceptrice.
%
%\subsection{Configurations}
%Il y a d'abord quelques étapes préliminaires :
%\begin{enumerate}
%	\item ouvrir l'Arduino IDE ;
%	\item brancher le module Arduino à votre PC via un câble USB ;
%	\item copier le code qui se trouve dans le fichier \textit{reception} fourni dans l'IDE ;
%	\item sauver le croquis (Sketch en anglais) en allant dans \textit{File --> Save} ou simplement en appuyant sur \textit{CTRL + S} ;
%\end{enumerate}
%
%Le résultat devrait être proche de celui de la Fig. \ref{fig:arduino_reception}.
%
%\begin{figure}[ht!]
%	\centering
%	\includegraphics[width=0.6\textwidth]{imgs/arduino_reception.png}
%	\caption{Exemple de code récepteur que nous vous fournissons.}
%	\label{fig:arduino_reception}
%\end{figure}
%
%Avant de programmer le module, il y a encore quelques configurations à effectuer :
%\begin{enumerate}
%	\setcounter{enumi}{3}
%	\item choisir la bonne "board", pour qu'elle corresponde à "Arduino Nano" ; pour cela, aller dans \textit{Tools --> Board --> Arduino Nano} (voir Fig. \ref{fig:arduino_board}) ;
%	\item choisir le bon processeur, pour qu'il corresponde à "ATmega328P" ; pour cela, aller dans \textit{Tools --> Processor --> ATmega328P} (voir Fig. \ref{fig:arduino_processor}) ;
%	\item choisir le bon port série ; pour cela, aller dans \textit{Tools --> Port} et choisir le port série adéquat (il n'y en a qu'un si vous n'avez qu'une seule connection USB) ;
%	\item voilà, il est temps de programmer votre module Arduino ; il suffit de cliquer sur \textit{Sketch --> Upload} ou d'utiliser \textit{CTRL + U}.
%\end{enumerate}
%
%\begin{figure}[ht!]
%	\centering
%	\includegraphics[width=0.6\textwidth]{imgs/arduino_board.png}
%	\caption{Configuration de la "board".}
%	\label{fig:arduino_board}
%\end{figure}
%
%\begin{figure}[ht!]
%	\centering
%	\includegraphics[width=0.6\textwidth]{imgs/arduino_processor.png}
%	\caption{Configuration du processeur.}
%	\label{fig:arduino_processor}
%\end{figure}
%
%Si tout se passe bien, votre carte électronique devrait faire de jolies choses !
%
%\newpage
%\subsection{Explication du code}
%Maintenant que votre carte est programmée, il est important de comprendre ce qu'elle fait. Pour cela, il suffit de comprendre le code repris à la Fig. \ref{fig:arduino_reception}. Le code est structuré en plusieurs parties, comme repris sur la figure :
%\begin{enumerate}
%	\item d'abord, on inclut les \textbf{librairies} utiles ; dans notre cas, la librairie "RF24", qui a besoin également de deux autres librairies ;
%	\item ensuite, on définit plusieurs \textbf{variables globales} ; par exemple, \texttt{LEDPIN = 3} indique que la pin 3 de l'Arduino est utilisée pour allumer les LEDs, via le LED driver ;
%	\item la fonction \texttt{setup} sert d'\textbf{initialisation} et n'est exécutée qu'une seule fois, au tout début (quand on branche l'alimentation par exemple) ; dans cette fonction :
%		\begin{itemize}
%			\item le moniteur série est configuré (ce dernier permet d'envoyer du texte à l'ordinateur ; pour l'activer, il suffit d'aller dans \textit{Tools --> Serial Monitor}) ;
%			\item la radio est configurée (canal, adresse, etc.) ;
%			\item les pins de sortie de l'Arduino sont configurées (LEDs et buzzer) ;
%		\end{itemize}
%	\item enfin, la fonction \texttt{loop} est la \textbf{partie centrale du code} ; cette fonction est exécutée indéfiniment après la phase d'initialisation ; dans le code d'exemple :
%		\begin{itemize}
%			\item on attend de recevoir un message et on le transmet vers l'ordinateur via le port série ;
%			\item on allume les LEDs via le LED driver ;
%			\item on allume la LED seule ou le buzzer pendant un petit moment.
%			\item et on recommence la fonction \texttt{loop} !
%		\end{itemize}
%\end{enumerate}
%
%\section{Conclusion}
%À partir de maintenant, vous pouvez partir du code exemple et le modifier à votre guise. Votre tâche est de recevoir correctement les messages émis par la balise émetrice pour pouvoir la localiser correctement.
%
%Voici quelques derniers conseils pour mener à bien votre mission:
%\begin{itemize}
%	\item nous vous avons fourni le code de l'émetteur (fichier \textit{transmit}) ; nous vous conseillons de bien le comprendre pour mettre en place votre stratégie de localisation !
%	\item quand vous utilisez les pins de l'Arduino, pour par exemple allumer une LED, vérifiez toujours que cette LED est bien connectée à la pin que vous choisissez dans le code.
%	\item posez des questions, n'hésitez surtout pas, nous sommes là pour cela !
%\end{itemize}
%
%\begin{center}
%Que la force soit avec vous !
%\end{center}

\end{document}
