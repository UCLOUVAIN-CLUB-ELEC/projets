Le but de cet exercice est d'établir une communication entre votre télécommande et l'Arduino. Typiquement, quand votre télécommande essaye de communiquer avec votre télévision, cela se passe 3 étapes:
\begin{enumerate}
  \item Quand vous appuyez sur un bouton, la télécommande envoie une valeur $X$
  via infrarouges.
  \item La télévision reçoit cette valeur $X$ via le récepteur infrarouges.
  \item En fonction de la valeur de $X$, le microcontrôleur de la télévision fait une action. Il peut
  soit augmenter le volume, changer de chaîne ou encore s'éteindre.
\end{enumerate}

La toute première étape est donc de savoir quelle valeur est envoyée par votre télécommande quand vous appuyez sur un certain bouton. Le programme \autoref{code1} permet de faire cela. En quelques mots, à chaque fois que l'Arduino reçoit un message infrarouges, il l'envoie directement à l'ordinateur. Cela vous permet donc de savoir le code correspondant à chacun des boutons de votre télécommande (notez ça sur un feuille ;)).

Pour voir les messages en console, vous devez aller dans \textit{Tools-->Serial Monitor}.

\lstset{language=C}
\begin{lstlisting}[frame=single,numbers=left,numberstyle=\small,label={code1},caption={Lecture de la télécommande}]
#include <IRremote.h>

const int RECV_PIN = 7;

IRrecv irrecv(RECV_PIN);
decode_results results;

void setup(){
  Serial.begin(9600);
  irrecv.enableIRIn();
}

void loop(){
  if (irrecv.decode(&results)){
        Serial.println(results.value, HEX);
        irrecv.resume();
  }
}
\end{lstlisting}
