Dans le hands-on 1, vous avez appris dans ce hands-on à contrôler des moteurs DC à partir de signaux de contrôle digitaux. Durant cette séance-ci, vous avez appris à envoyer des commandes à distances à un module Arduino équipé d'un récepteur infrarouges, grâce à une télécommande. Vous êtes maintenant capables de combiner les deux hands-on et de créer un petit robot à téléguider. Nous vous suggérons d'utiliser la fonction \texttt{analogWrite} qui est très utile car elle fait de la modulation à largeur d'impulsions (MLI) pour vous ! Pensez-y pour régler la vitesse de votre robot. Pour plus d'informations sur la fonction \texttt{analogWrite}, vous pouvez consulter le lien suivant: \url{www.arduino.cc/reference/en/language/functions/analog-io/analogwrite/}.
