Une fois que vous avez identifié le code que votre télécommande envoie pour un certain bouton, vous pouvez adapter l’exemple \autoref{code2}.
Ici, à chaque itération de la fonction \textit{loop()}, l'Arduino regarde si le récepteur infrarouges a reçu une message. Pour chaque message reçu, il regarde à quel bouton ce code correspond (ici $AB456CD$ pour le bouton 1 et $05FBAC33$ pour le bouton 2). Ensuite, il exécute une fonction correspondant au bouton utilisé (\textit{IR\_button1()}).
Dans le cas du bouton 1, la Led s'allume ou s'éteint à chaque fois que le bouton est appuyé.

\lstset{language=C}
\begin{lstlisting}[frame=single,numbers=left,numberstyle=\small,label={code2},caption={Code 2}]
  #include <IRremote.h>

  #define REMOTE_BUTTON1    0xAB456CD   // Ligne a modifier
  #define REMOTE_BUTTON2    0x5FBAC33   // Ligne a modifier

  const int RECV_PIN = 7;
  const int LED_PIN = 13;

  IRrecv irrecv(RECV_PIN);
  decode_results results;

  int LED_val = 0;

  void setup() {
    Serial.begin(9600);
    pinMode(LED_PIN,OUTPUT);
    digitalWrite(LED_PIN,LED_val);
    irrecv.enableIRIn();
    delay(500);
  }

  void loop() {
    if (irrecv.decode(&results)){
      //Serial.println(results.value, HEX);
      irrecv.resume();
      switch (results.value) {
        case REMOTE_BUTTON1:
          IR_button1();
          break;
        case REMOTE_BUTTON2:
          IR_button2();
          break;
      }
      irrecv.resume();
      }
  }

  void IR_button1(){
      LED_val = 1-LED_val;
      digitalWrite(LED_PIN,LED_val);
  }

  void IR_button2(){
    // faire qqch
  }
\end{lstlisting}
