Vous avez appris dans ce hands-on à contrôler des moteurs DC à partir de signaux de contrôle digitaux. Dans votre circuit, les valeurs de ces signaux de contrôle étaient fixées manuellement. Pour la suite de ce projet, vous recevrez, entre autres, un microcontrôleur de type Arduino qui permet d'envoyer des signaux de contrôle 5V, ainsi que 2 moteurs DC, un L293D et une pile 9V. Imaginez dès à présent la stratégie la plus efficace pour contrôler les mouvements de votre robot à partir de ce matériel. Voici quelques questions pour vous aiguiller:
\begin{itemize}
\item Quels sont les signaux à envoyer au L293D pour avancer? Et pour reculer?
\item Comment contrôler chacun des moteurs pour tourner?
\item Comment faire pour régler la vitesse du robot? 
\end{itemize}