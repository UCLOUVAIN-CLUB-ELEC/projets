\subsubsection*{Organisation}
\begin{itemize}
\item Les groupes qui ont participé au Projet Robot du Club Elec peuvent participer au concours avec le robot qu'ils ont construit.
\item Le concours est divisé en 2 courses. Il est possible de participer à l'une et/ou à l'autre indépendamment.
\item Les groupes passent un à la fois dans les parcours.
\item Chaque groupe à droit à 1 essai pour chacune des courses.
\item Le robot peut être modifié ou reprogrammé entre les 2 courses.
\item La course téléguidée est une course contre la montre à travers un parcours d'obstacles durant laquelle un pilote humain peut interagir avec le robot.
\item Il est interdit de passer au dessus des obstacles. 
\item Le groupe gagnant de la course téléguidée est celui dont le robot arrivera au bout du parcours avec le temps le plus court.
\item La course autonome est une course durant laquelle le robot doit faire un maximum de tours le long d'une boucle tracée au sol en 2 minutes, sans intervention humaine.
\item Le groupe peut interagir avec le robot avant le début de la course tant que le robot est sur la ligne de départ.
\item Le groupe gagnant de la course autonome est celui dont le robot aura effectué la plus grande distance durant le temps imparti.
\item Un exemple de boucle (différent de celle utilisée lors du concours) sera fourni pour l'entrainement pour la course autonome.
\end{itemize}

\subsubsection*{Spécifications}
\begin{itemize}
\item La projection du robot sur le sol doit être comprise dans un rectangle de 15$\times$20 cm.
\item Le robot est alimenté par une pile de 9V.
\item Le contrôle du robot se fait par l'Arduino Nano.
\item Le robot est propulsé par 2 moteurs électriques DC au maximum.
\item Pour la course chronométrée, la ligne est de couleur noire sur fond blanc avec une largeur de 2~cm.
\end{itemize}