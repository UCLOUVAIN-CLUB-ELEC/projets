% Principe de la transduction du signal sonore, du micro (electret microphone dans ce cas-ci), gamme de fréquence + schéma uniquement micro+résistance

Lorsque quelqu'un claque des doigts, une onde sonore est produite, qui consiste en fait en une variation locale de la pression de l'air qui se propage dans l'espace jusqu'à nos oreilles. Afin de commander un circuit électrique à partir de ce claquement de doigt, il est donc nécessaire de pouvoir transformer ce changement de pression en une variation d'une quantité électrique qu'il est possible de mesurer. Cette transformation d'une grandeur physique en une caractéristique électrique s'appelle la transduction. Dans le cas d'un signal sonore, cette transduction se fait à l'aide d'un dispositif bien connu: le microphone. \\

Il existe différents types de microphones (à ruban, à condensateur...) avec des fonctionnements, des performances et des prix différents. Celui mis à votre disposition pour ce projet est le modèle \texttt{ABM-707-RC}. Il s'agit d'un micro électrostatique à électret. C'est un micro qui s'adapte assez bien à un petit projet d'électronique au vu de sa petite taille, de son prix très démocratique et de sa grande facilité d'utilisation. \\

Un micro à électret se compose d'une membrane polarisée, c'est-à-dire qu'elle porte une charge électrostatique, d'une électrode et d'un circuit d'amplification. Lorsqu'une onde sonore arrive, elle va faire vibrer la membrane à l'intérieur du micro, ce qui va faire varier sa distance par rapport à l'électrode. Grâce au champ électrique généré par la membrane polarisée, un changement de tension va être produit sur l'électrode. Ce changement de tension sera ensuite amplifié par l'amplificateur avant de se retrouver à la sortie du micro.\\

Voici brièvement quelques caractéristiques techniques du micro utilisé:
\begin{itemize}
\item Ce micro nécessite une alimentation allant de 2 à 10V, utilisée pour son amplificateur interne. Dans ce projet, nous utiliserons une tension de 5V.
\item Ce micro couvre une plage de fréquences allant de 50 Hz à 16kHz, ce qui est largement suffisant pour l'application visée. Pour votre information, les humains perçoivent des fréquences allant (grossièrement) de 20 à 20 kHz.
\item Ce micro a une sensitivité de -41dB.
\item Ce micro est omnidirectionnel, c'est-à-dire qu'il entend aussi bien dans toutes les directions.
\end{itemize}

