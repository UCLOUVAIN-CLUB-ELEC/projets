% Objectifs: prise en main du micro et du signal sonore, notion AC/DC, utilisation de l'oscilloscope, découplage AC

Les objectifs du premier hands-on sont:

\begin{itemize}
	\item[-] De se familiariser avec le matérial de base (breadboard, multimètre, oscilloscope) et les composants de base (résistances, capacités, amplificateurs opérationnels, composants intégrés) propres à l'électronique.
	\item[-] De comprendre le fonctionnement du micro qui assure la transduction du signal sonore en signal électrique.
	\item[-] De faire le lien entre le signal obtenu et son contenu fréquentiel afin de comprendre la notion de filtrage.
	\item[-] De comprendre la notion AC/DC et le découplage AC.
	\item[-] D'implémenter en pratique la première partie du circuit (micro, filtre et découplage).
\end{itemize}

\begin{figure}[!ht]
	\centering
	\includegraphics[width=.75\textwidth]{figures/SchemaBloc.pdf}
	\caption{Schéma-bloc du circuit.}
	\label{fig:block-diagram}
\end{figure}

Le schéma-bloc du circuit est présenté à la Figure \ref{fig:block-diagram}. Les ondes acoustiques générées par le claquement de doigt sont captées par le micro qui les transforme en un signal électrique (transduction). Ce signal est ensuite découplé et filtré à l'aide d'un filtre RC, comme présenté plus loin dans ce document. La génération d'un signal de commande propre ainsi que l'implémentation d'un circuit commandé seront abordées plus en détail dans les prochain hands-on.
